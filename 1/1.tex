%%%%%%%%%%%%%%%%%%%%%%%%%%%%%%%%%%%%%%%%%%%%%%%%
%%   TEMPLATE MAIN FILE                       %%
%%%%%%%%%%%%%%%%%%%%%%%%%%%%%%%%%%%%%%%%%%%%%%%%

\documentclass[10pt,reqno,sumlimits]{amsart}
\usepackage{amssymb}
\usepackage{epsfig}
\usepackage{multirow}
\usepackage{graphics}
\usepackage{graphicx}
\usepackage{subfigure}
\usepackage{url}
%\textwidth 6.2in
%\oddsidemargin.20in
%\evensidemargin.35in
%%\lineskip 1in
%%\baselineskip.55cm

% Read in specially defined commands
% Change margins and baselinestretch in draft mode
\def\draft{
% make margins smaller
\addtolength{\oddsidemargin}{-0.5in}
\addtolength{\topmargin}{-0.65in}
\addtolength{\textheight}{1in}
\addtolength{\textwidth}{1.5in}
% A more reasonable baselinestretch, is still nearly doublespaced.
\def\baselinestretch{1.4}
}

%\vfuzz2pt % Don't report over-full v-boxes if over-edge is small
%\hfuzz2pt % Don't report over-full h-boxes if over-edge is small

%%%%%%%%%%%%%%%%%%%%%%%%%%%%%%%%%%%%%%%%%%%%%%%%%%%%%%%%
%%              Theorems                               %
%%%%%%%%%%%%%%%%%%%%%%%%%%%%%%%%%%%%%%%%%%%%%%%%%%%%%%%%
\theoremstyle{plain}
\newtheorem{theorem}{Theorem}
\newtheorem{corollary}[theorem]{Corollary}
\newtheorem{lemma}[theorem]{Lemma}
\newtheorem{proposition}[theorem]{Proposition}
\newtheorem{conjecture}[theorem]{Conjecture}

\theoremstyle{definition}
\newtheorem{remark}[theorem]{Remark}
\newtheorem{definition}[theorem]{Definition}
\newtheorem{example}[theorem]{Example}
\newtheorem{notation}{Notation}

%%%%%%%%%%%%%%%%%%%%%%%%%%%%%%%%%%%%%%%%%%%%%%%%%%%%%%%%
%%      Definitions and Commands                       %
%%%%%%%%%%%%%%%%%%%%%%%%%%%%%%%%%%%%%%%%%%%%%%%%%%%%%%%%
\newcommand{\A}{{\mathbb A}}
\newcommand{\R}{{\mathbb R}}
\newcommand{\Q}{{\mathbb Q}}
\newcommand{\C}{{\mathbb C}}
\newcommand{\D}{{\mathbb D}}
\newcommand{\Z}{{\mathbb Z}}
\newcommand{\h}{{\mathbb H}}
\newcommand{\CP}{{\mathbb C}{\mathbb P}}
\newcommand{\I}{{\mathbb I}}
\newcommand{\N}{{\mathbb N}}
%
\newcommand{\calA}{{\mathcal A}}
\newcommand{\calB}{{\mathcal B}}
\newcommand{\calC}{{\mathcal C}}
\newcommand{\calD}{{\mathcal D}}
\newcommand{\calE}{{\mathcal E}}
\newcommand{\calF}{{\mathcal F}}
\newcommand{\calG}{{\mathcal G}}
\newcommand{\calH}{{\mathcal H}}
\newcommand{\calI}{{\mathcal I}}
\newcommand{\calK}{{\mathcal K}}
\newcommand{\calM}{{\mathcal M}}
\newcommand{\calO}{{\mathcal O}}
\newcommand{\calP}{{\mathcal P}}
\newcommand{\calR}{{\mathcal R}}
\newcommand{\calS}{{\mathcal S}}
\newcommand{\calL}{{\mathcal L}}
\newcommand{\calX}{{\mathcal X}}
\newcommand{\calU}{{\mathcal U}}
\newcommand{\calV}{{\mathcal V}}
\newcommand{\calZ}{{\mathcal Z}}
%
\newcommand{\ggoth}{{\mathfrak  g}}
\newcommand{\ugoth}{{\mathfrak  u}}
\newcommand{\hgoth}{{\mathfrak  h}}
%%
\newcommand{\1}{{\bf 1}}
\newcommand{\acts}{{\circlearrowright}}
\newcommand{\ex}[1]{{e^{#1}}}
\newcommand{\dd}[2]{\frac{\partial #1}{\partial #2}}
\newcommand{\iso}{\stackrel{\simeq}{\longrightarrow}}
\newcommand{\spinc}{$\text{spin}^c$}
\newcommand{\Dirac}{\not\!\!D}
\newcommand{\inc}{\hookrightarrow}
%%
\newcommand{\Aut}{\operatorname{Aut}}
\newcommand{\Hom}{\operatorname{Hom}}
\newcommand{\Hor}{\operatorname{Hor}}
\newcommand{\Ext}{\operatorname{Ext}}
\newcommand{\End}{\operatorname{End}}
\newcommand{\Map}{\operatorname{Map}}
\newcommand{\Diff}{\operatorname{Diff}}
\newcommand{\Tr}{\operatorname{Tr}}
\newcommand{\Lie}{\operatorname{Lie}}
\newcommand{\ad}{{\operatorname{ad\,}}}
\newcommand{\sign}{{\operatorname{sign}}}
\newcommand{\grad}{{\operatorname{grad}}}
\newcommand{\coker}{{\operatorname{coker}}}
\newcommand{\dett}{{\operatorname{det}}}
\newcommand{\ch}{{\operatorname{ch}}}
\newcommand{\rk}{{\operatorname{rk}}}
\newcommand{\maxx}{{\operatorname{max}}}
\newcommand{\minn}{{\operatorname{min}}}
\newcommand{\id}{{\operatorname{id}}}
\newcommand{\ind}{{\operatorname{ind}}}
\newcommand{\Ind}{{\operatorname{Ind}}}
\newcommand{\spann}{{\operatorname{span}}}
\newcommand{\Spin}{{\operatorname{Spin}}}
\newcommand{\Pin}{{\operatorname{Pin}}}
\newcommand{\im}{{\operatorname{Im}}}
\renewcommand{\Im}{{\operatorname{Im}}}
\newcommand{\dimm}{{\operatorname{dim}}}
\newcommand{\cl}{{\operatorname{cl}}}
%%
\newcommand{\tM}{{\tilde{M}}}
\newcommand{\tC}{{\tilde{C}}}
\newcommand{\tE}{{\tilde{E}}}
\newcommand{\tF}{{\tilde{F}}}
\newcommand{\talpha}{{\tilde{\alpha}}}
%%
\newcommand{\zbar}{\overline{z}}
\newcommand{\wbar}{\overline{w}}
\newcommand{\phibar}{\overline{\phi}}
\newcommand{\psibar}{\overline{\psi}}
\newcommand{\Bbar}{\overline{B}}
\newcommand{\Cbar}{\overline{C}}
\newcommand{\inv}{^{-1}}
%%      
\newcommand{\rb}[1]{\raisebox{0pt}[0pt][0pt]{#1}}
\newsavebox{\savepar}
\newenvironment{boxit}{\begin{lrbox}{\savepar}\begin{minipage}[b]{.5in}}
{\end{minipage}\end{lrbox}\fbox{\usebox{\savepar}}}
%%\newcommand{\ip}[1]{\langle\! #1 \! \rangle}
\newcommand{\ip}[1]{\langle #1 \rangle}
\newcommand{\norm}[1]{\| #1 \|}
%
%\newcommand{\remark}{{\tt ?}\marginpar{\Large\centering ?}}
\newcommand{\rremark}{\marginpar{\Large\centering ?}}
\newcommand{\Remark}[1]{\textsc{\tiny{[#1]}}\rremark}
\newcommand{\todo}{\textsc{Todo!}\marginpar{\Large\centering !}}
%
\numberwithin{equation}{section}
\renewcommand{\theequation}{{\thesection{.}}\arabic{equation}}
\newcounter{dummy}
%
%%%%%%%%%%%%%%%%%%%%%%%%%%%%%%%%%%%%%%%%%%%%%%%%%%%%%%%
\begin{document}

\title[Assignment 1]{The Beginnings of Algorithms}
\author{Joe Smith}


%\begin{abstract}
%The abstract
%\end{abstract}

\maketitle

%\tableofcontents

%%%%%%%%%%%%%%%%%%%%%%%%%%%%%%%%%%%%%%%%%%%%%%%%%%%%%%%

\section {How slow can they go?}
How much slower do each of these algorithms get when you (a) double the input sizr, or (b) increase the input size by one?

\begin{enumerate}
% (a)
\item $n^2$
\begin{enumerate}
% a
\item $(2n)^2$

\hspace{0.3in}$= 4n^2 \rightarrow$ 4 times slower.
% b
\item $(n+1)^2$

\hspace{0.3in}$= n^2 + 2n + 1\rightarrow $ slower by 2n + 1.
\end{enumerate}

% (b)
\item $n^3$
\begin{enumerate}
% a
\item $(2n)^3$

\hspace{0.3in}$= 8n^3\rightarrow $ 8 times slower.
% b
\item $(n+1)^3$

\hspace{0.3in}$= n^3 + 3n^2 + 3n + 1\rightarrow $ slower by 3n$^2$ + 3n + 1
\end{enumerate}

% (c)
\item $100n^2$
\begin{enumerate}
% a
\item $100(2n)^2$

\hspace{0.3in}$= 100(4)n^2\rightarrow $ 4 times slower.
% b
\item $100(n+1)^2$

\hspace{0.3in}$= 100n^2 + 200n + 100\rightarrow $ slower by 200n + 100.
\end{enumerate}

% (d)
\item $n\log n$
\begin{enumerate}
% a
\item $(2n)\log(2n)$

\hspace{0.3in}$= n\log(2n) + n\log(2n) =$

\hspace{0.4in}$= n\log(n) + n\log(2) + n\log(2n)$

\hspace{0.4in}$\rightarrow$ slower by $n\log(2) + n\log(2n)$

% b
\item $(n+1)\log(n+1)$

\hspace{0.3in}$\approx (n+1)\log(n) \rightarrow $ slower by $\log(n)$.

I don't quite buy this one, but supposedly $\log(n+1)$ is unbreakable.
\end{enumerate}

% (e)
\item $2^n$
\begin{enumerate}
% a
\item $2^(2n)$

\hspace{0.3in}$= 2^n \cdot 2^n \rightarrow $ $2^n$ times slower.
% b
\item $2^(n+1)$

\hspace{0.3in}$= 2^n \cdot 2^1 \rightarrow $ 2 times slower.
\end{enumerate}
\end{enumerate}

\section{Large and In Charge}
\begin{enumerate}
\item $n^2$

$ n^2 = 10^{10} \cdot 3600$

$ n = \sqrt{10^{10} \cdot 3600}$

$ n = 6,000,000$

\item $n^3$

$ 33019 $

\item $100n^2$

$ \sqrt{ (10^{10} \cdot 3600) / 100}

\item $n\log n$

\item $2^n$

\item $2^{2^n}$
\end{enumerate}
%\begin{theorem}
%Suppose $I$ is an open interval on $\R$, and $f:I\to\R$ is differentiable at $a\in I$. Then $f$ is continuous at $a$. Moreover, if $f$ is differentiable on $I$, then $f$ is continuous on $I$.
%\end{theorem}

%\begin{proof}
%Choose arbitrarily $a\in I$. We have to show that $f(x)\to f(a)$, when $x\to a$.

%First, if $x\in I, \quad x\neq a$, then
%$$
%f(x)-f(a) = \frac{f(x)-f(a)}{x-a}\cdot (x-a).
%$$
%Thus, if $f'(a)$ is the derivative of $f$ at $a$, we have

%\begin{eqnarray}
%\lim_{x \to a} (f(x) - f(a)) &=& \lim_{x \to a} \frac{f(x) - f(a)}{x-a} \cdot (x-a)\nonumber\\
%&=& \lim_{x \to a} \frac{f(x) - f(a)}{x-a} \cdot \lim_{x \to a}(x-a)\nonumber\\
%&=& f'(a) \cdot 0 = 0,\nonumber
%\end{eqnarray}
%\end{proof}

\end{document}

%%%%%%%%%%%%%%%%%%%%%%%%%%%%%%%%%%%%%%%%%%%%%%%%%%%%%%%

%%% Local Variables: 
%%% mode: latex
%%% TeX-master: "main"
%%% End: 

