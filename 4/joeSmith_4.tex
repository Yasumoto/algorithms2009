%%%%%%%%%%%%%%%%%%%%%%%%%%%%%%%%%%%%%%%%%%%%%%%%
%%   TEMPLATE MAIN FILE                       %%
%%%%%%%%%%%%%%%%%%%%%%%%%%%%%%%%%%%%%%%%%%%%%%%%

\documentclass[10pt,reqno,sumlimits]{amsart}
\usepackage{amssymb}
\usepackage{epsfig}
\usepackage{multirow}
\usepackage{graphics}
\usepackage{graphicx}
\usepackage{subfigure}
\usepackage{url}
%\textwidth 6.2in
%\oddsidemargin.20in
%\evensidemargin.35in
%%\lineskip 1in
%%\baselineskip.55cm

% Read in specially defined commands
% Change margins and baselinestretch in draft mode
\def\draft{
% make margins smaller
\addtolength{\oddsidemargin}{-0.5in}
\addtolength{\topmargin}{-0.65in}
\addtolength{\textheight}{1in}
\addtolength{\textwidth}{1.5in}
% A more reasonable baselinestretch, is still nearly doublespaced.
\def\baselinestretch{1.4}
}

%\vfuzz2pt % Don't report over-full v-boxes if over-edge is small
%\hfuzz2pt % Don't report over-full h-boxes if over-edge is small

%%%%%%%%%%%%%%%%%%%%%%%%%%%%%%%%%%%%%%%%%%%%%%%%%%%%%%%%
%%              Theorems                               %
%%%%%%%%%%%%%%%%%%%%%%%%%%%%%%%%%%%%%%%%%%%%%%%%%%%%%%%%
\theoremstyle{plain}
\newtheorem{theorem}{Theorem}
\newtheorem{corollary}[theorem]{Corollary}
\newtheorem{lemma}[theorem]{Lemma}
\newtheorem{proposition}[theorem]{Proposition}
\newtheorem{conjecture}[theorem]{Conjecture}

\theoremstyle{definition}
\newtheorem{remark}[theorem]{Remark}
\newtheorem{definition}[theorem]{Definition}
\newtheorem{example}[theorem]{Example}
\newtheorem{notation}{Notation}

%%%%%%%%%%%%%%%%%%%%%%%%%%%%%%%%%%%%%%%%%%%%%%%%%%%%%%%%
%%      Definitions and Commands                       %
%%%%%%%%%%%%%%%%%%%%%%%%%%%%%%%%%%%%%%%%%%%%%%%%%%%%%%%%
\newcommand{\A}{{\mathbb A}}
\newcommand{\R}{{\mathbb R}}
\newcommand{\Q}{{\mathbb Q}}
\newcommand{\C}{{\mathbb C}}
\newcommand{\D}{{\mathbb D}}
\newcommand{\Z}{{\mathbb Z}}
\newcommand{\h}{{\mathbb H}}
\newcommand{\CP}{{\mathbb C}{\mathbb P}}
\newcommand{\I}{{\mathbb I}}
\newcommand{\N}{{\mathbb N}}
%
\newcommand{\calA}{{\mathcal A}}
\newcommand{\calB}{{\mathcal B}}
\newcommand{\calC}{{\mathcal C}}
\newcommand{\calD}{{\mathcal D}}
\newcommand{\calE}{{\mathcal E}}
\newcommand{\calF}{{\mathcal F}}
\newcommand{\calG}{{\mathcal G}}
\newcommand{\calH}{{\mathcal H}}
\newcommand{\calI}{{\mathcal I}}
\newcommand{\calK}{{\mathcal K}}
\newcommand{\calM}{{\mathcal M}}
\newcommand{\calO}{{\mathcal O}}
\newcommand{\calP}{{\mathcal P}}
\newcommand{\calR}{{\mathcal R}}
\newcommand{\calS}{{\mathcal S}}
\newcommand{\calL}{{\mathcal L}}
\newcommand{\calX}{{\mathcal X}}
\newcommand{\calU}{{\mathcal U}}
\newcommand{\calV}{{\mathcal V}}
\newcommand{\calZ}{{\mathcal Z}}
%
\newcommand{\ggoth}{{\mathfrak  g}}
\newcommand{\ugoth}{{\mathfrak  u}}
\newcommand{\hgoth}{{\mathfrak  h}}
%%
\newcommand{\1}{{\bf 1}}
\newcommand{\acts}{{\circlearrowright}}
\newcommand{\ex}[1]{{e^{#1}}}
\newcommand{\dd}[2]{\frac{\partial #1}{\partial #2}}
\newcommand{\iso}{\stackrel{\simeq}{\longrightarrow}}
\newcommand{\spinc}{$\text{spin}^c$}
\newcommand{\Dirac}{\not\!\!D}
\newcommand{\inc}{\hookrightarrow}
%%
\newcommand{\Aut}{\operatorname{Aut}}
\newcommand{\Hom}{\operatorname{Hom}}
\newcommand{\Hor}{\operatorname{Hor}}
\newcommand{\Ext}{\operatorname{Ext}}
\newcommand{\End}{\operatorname{End}}
\newcommand{\Map}{\operatorname{Map}}
\newcommand{\Diff}{\operatorname{Diff}}
\newcommand{\Tr}{\operatorname{Tr}}
\newcommand{\Lie}{\operatorname{Lie}}
\newcommand{\ad}{{\operatorname{ad\,}}}
\newcommand{\sign}{{\operatorname{sign}}}
\newcommand{\grad}{{\operatorname{grad}}}
\newcommand{\coker}{{\operatorname{coker}}}
\newcommand{\dett}{{\operatorname{det}}}
\newcommand{\ch}{{\operatorname{ch}}}
\newcommand{\rk}{{\operatorname{rk}}}
\newcommand{\maxx}{{\operatorname{max}}}
\newcommand{\minn}{{\operatorname{min}}}
\newcommand{\id}{{\operatorname{id}}}
\newcommand{\ind}{{\operatorname{ind}}}
\newcommand{\Ind}{{\operatorname{Ind}}}
\newcommand{\spann}{{\operatorname{span}}}
\newcommand{\Spin}{{\operatorname{Spin}}}
\newcommand{\Pin}{{\operatorname{Pin}}}
\newcommand{\im}{{\operatorname{Im}}}
\renewcommand{\Im}{{\operatorname{Im}}}
\newcommand{\dimm}{{\operatorname{dim}}}
\newcommand{\cl}{{\operatorname{cl}}}
%%
\newcommand{\tM}{{\tilde{M}}}
\newcommand{\tC}{{\tilde{C}}}
\newcommand{\tE}{{\tilde{E}}}
\newcommand{\tF}{{\tilde{F}}}
\newcommand{\talpha}{{\tilde{\alpha}}}
%%
\newcommand{\zbar}{\overline{z}}
\newcommand{\wbar}{\overline{w}}
\newcommand{\phibar}{\overline{\phi}}
\newcommand{\psibar}{\overline{\psi}}
\newcommand{\Bbar}{\overline{B}}
\newcommand{\Cbar}{\overline{C}}
\newcommand{\inv}{^{-1}}
%%      
\newcommand{\rb}[1]{\raisebox{0pt}[0pt][0pt]{#1}}
\newsavebox{\savepar}
\newenvironment{boxit}{\begin{lrbox}{\savepar}\begin{minipage}[b]{.5in}}
{\end{minipage}\end{lrbox}\fbox{\usebox{\savepar}}}
%%\newcommand{\ip}[1]{\langle\! #1 \! \rangle}
\newcommand{\ip}[1]{\langle #1 \rangle}
\newcommand{\norm}[1]{\| #1 \|}
%
%\newcommand{\remark}{{\tt ?}\marginpar{\Large\centering ?}}
\newcommand{\rremark}{\marginpar{\Large\centering ?}}
\newcommand{\Remark}[1]{\textsc{\tiny{[#1]}}\rremark}
\newcommand{\todo}{\textsc{Todo!}\marginpar{\Large\centering !}}
%
\numberwithin{equation}{section}
\renewcommand{\theequation}{{\thesection{.}}\arabic{equation}}
\newcounter{dummy}
%
%%%%%%%%%%%%%%%%%%%%%%%%%%%%%%%%%%%%%%%%%%%%%%%%%%%%%%%
\begin{document}

\title[Assignment 4]{Completely NP}
\author{Joe Smith}


%\begin{abstract}
%The abstract
%\end{abstract}

\maketitle

%\tableofcontents

%%%%%%%%%%%%%%%%%%%%%%%%%%%%%%%%%%%%%%%%%%%%%%%%%%%%%%%

\section {Vertex Cover}
In the vertex covering problem we are given a graph $G$ consisting of a set of vertices $V$ and edges $E$. We would like to determine if for a given integer $k$ there is a subset $V'$ of $V$ with the size of $V'$ less than $k$ such that every edge has at least one endpoint in $V'$. 
\\
Prove that vertex cover is NP complete by showing 3-SAT is poly nomial time reducible to it.
\subsection{Answer}
Given a problem of 3-SAT, with clauses $C_1, C_2, ..., C_n$ and boolean variables $x_1, x_2, ...\ , x_n$, we can reduce it to a problem of vertex cover. For each clause, $C_x$, it contains three boolean variables, $x_a, x_b, x_c$. Make each variable a node in a graph, and create edges between all three nodes. For each variable present in the equation, $x_1, x_2, ...\ , x_n$, place a node for that variable as well as its disjunction. Connect each of these nodes with its disjunction, such that the $x_1$ node is connected to each $\bar{x_1}$ present in a clause, and the $\bar{x_1}$ node is connected to each $x_1$ node in each clause. Thus, each clause that is true will have its variables true, and the nodes will be included in the subgraph for vertex cover.
\\


\section {Monotone Satisfiability}
Given a monotone instance of Satisfiability, together with a number $k$, the problem of \textit{Monotone Satisfiability with Few True Variables} asks: Is there a satisfying assignment for the instance in which at most $k$ variables are set to 1? Prove this problem is NP-complete.
\subsection{Answer}
In order to show this problem is NP-Complete, we will reduce vertex cover to this problem. A graph can be converted to Monotone Satisfiability if there is a graph such that each clause is an edge. This edge will be composed of two nodes, each of which will be variables featured in a clause. Vertex cover states that each edge in the graph will have a connected vertex in the subgraph. As such, if a variable is considered true in Monotone Satisfiability, then it will be included in the subgraph, and will satisfy vertex cover.



\end{document}

%%%%%%%%%%%%%%%%%%%%%%%%%%%%%%%%%%%%%%%%%%%%%%%%%%%%%%%

%%% Local Variables: 
%%% mode: latex
%%% TeX-master: "main"
%%% End: 

